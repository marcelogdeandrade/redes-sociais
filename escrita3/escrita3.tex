



\documentclass[12pt]{article}

\usepackage[brazil]{babel}
\usepackage[utf8]{inputenc}
\usepackage{graphicx}
\usepackage{float}
\usepackage{indentfirst}

\title{Rede de julgamentos entre professores e análise de seu comportamento a longo prazo}
\author{
	Marcelo Gianfaldoni de Andrade\\
	Engenharia da Computação\\
	Insper\\
}
\date{\today}


\begin{document}
\maketitle

\newpage
\tableofcontents
\newpage

\begin{abstract}
Esse texto busca analisar as relações entre professores após um final de semana de "retreat", entender como eles se conectam em termos de julgamentos, e quais comportamentos entre eles podem ser mais adequados para diferentes consequências: a coesão total entre eles, a polarização dos professores em dois grupos distintos, e uma maior fragmentação desse grupo de professores.
\end{abstract}

\section{Introdução}
Após um final de semana juntos, um grupo de professores desenvolveu uma série de relações entre si. Para o objetivo final desse ``retreat" do final de semana, o desenvolvimento de novas disciplinas, a relação mais importante a ser analisada é o julgamento entre eles. Dois professores podem gostar-se mutuamente, criando uma relação positiva, ou não gostar-se mutuamente, criando uma relação negativa. Esse trabalho irá analisar as possíveis dinâmicas internas dessas relações com o tempo a partir de três situações distintas que alteram essas conexões, para enfim, concluir quais são as consequências possíveis de diferentes atuações dessas situações nas relações entre os professores.

\section{Situações e Consequências}
Como explicado acima, há três situações distintas que mudam as relações entre os professores após o final de semana: 

\begin{itemize}
	\item Inimigos em Comum
	\item Pacificador
	\item Cizânia
\end{itemize}

A primeira situação, inimigos em comum, acontece quando três professores não se gostam mutuamente, portanto, espera-se que uma ligação positiva aconteça entre dois deles, pois têm o terceiro como inimigo em comum. 

A segunda situação, pacificador, acontece quando um professor gosta mutuamente de dois outros, mas esses dois não se gostam mutuamente, espera-se que esse primeiro professor seja um pacificador entre os dois outros, criando uma relação positiva entre eles.

A terceira e última situação, cizânia, acontece quando um professor gosta de um outro, e não gosta de um terceiro, mas esses outros dois professores se gostam mutuamente. Espera-se que esse professor convença o outro a parar de gostar do terceiro, criando assim uma relação negativa entre eles.

Os diferentes ``pesos" para a combinação dessas três situações gera uma rede de relações de julgamentos diferente para os professores a longo prazo. Dos diferentes resultados possíveis, pode-se destacar três: A total coesão, em que todos os professores tem relações positivas, portanto formam um só grupo conjunto. A polarização, em que os professores acabam se dividindo em dois grupos distintos de conexões positivas, e ligados entre si de modo negativo. E por fim, a fragmentação dos professores, em que são formados diferentes grupos menos polarizados. 

\section{Análise}

Primeiramente, pode-se dividir as situações em dois grupos: Aquelas que geram ligações positivas(inimigos em comum e pacificador), e quem geram ligações negativas(cizânia). Caso ocorram apenas situações que geram ligações positivas, o resultado final a longo prazo será um grupo coeso, pois são geradas apenas ligações positivas.

Observado essa mesma divisão, caso ocorram apenas situações que geram ligações negativas(cizânia), o grupo será cada vez mais fragmentado, pois a longo prazo serão geradas somente ligações negativas entre os professores.

Por fim, para entender como chega-se em uma consequência polarizadora, pode-se dividir as situações que geram ligações positivas entre uma situação polarizadora(inimigos em comum), por criar uma relação positiva e isolar o terceiro professor, e uma situação coesa(pacificador), por gerar um grupo coeso entre três professores. Dessa forma, caso ocorra a situação de inimigos em comum de modo conjunto a situação de cizânia, é esperado que o resultado a longo prazo seja um grupo de professores polarizado.

\section{Conclusão}

Após entender que diferentes tipos de situações têm diferentes consequências na rede a longo prazo, pode-se perceber quais são mais importantes para um resultado esperado. No caso do desenvolvimento de disciplinas em conjunto, um grupo coeso é o esperado para os professores, portanto, é recomendado evitar a cizânia entre eles, assim as duas outras situações geram um grupo coeso.

\end{document}
  